\begin{aufgabe}
Zu Beginn steht der Zeiger auf der ersten Stelle der Eingabe und die TM befindet sich im Zustand $q_0$. Nach dem Lesen einer 1 auf der ersten Stelle
bewegt sich der Zeiger eine Stelle nach rechts, verweilt aber im Zustand $q_0$. Auf der zweiten Position steht auch eine 1, deshalb wird der Zeiger 
einen Schritt nach rechts auf die dritte Position bewegt und verharrt weiter in Zustand $q_0$. Dort wird eine 0 gelesen, deshalb wird der Zeiger wieder
eine Stelle nach rechts bewegt und bleibt im Zustand $q_0$. Dort steht ein Blank, der Zustand wechselt also von $q_0$ zu $q_1$ und der Zeiger wird
eine Stelle nach links bewegt. Dort wird eine 0 gelesen, der Zeiger also in auf die vierte Stelle gelegt und der Zustand wechselt in den Endzustand $\bar{q}$.


\end{aufgabe}